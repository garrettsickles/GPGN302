\section{Introduction}

%% 
 % SNR improvement by stacking 
 %%
Redundant seismic experiments are designed to increase the data
signal-to-noise (SNR) ratio. SNR is improved when traces from
different experiments are summed -- the coherent signal amplifies, but
the incoherent noise attenuates. 

We can use stack, for example, in common-midpoint gathers (CMP),
although this is not the only possibility. Under the assumption that
the medium is laterally invariant, i.e. the reflectors are horizontal
and the velocity does not change laterally, all traces contained in a
CMP sample the same points in the subsurface located right under the
surface common midpoint. In this case, the signal from traces
corresponding to different offsets is coherent, while the noise
contained in each trace is not.

% ------------------------------------------------------------
\inputdir{XFig}
% ------------------------------------------------------------
\sideplot{cmp}{width=\textwidth}{Common-midpoint gather geometry
  highlighting repeat measurements of the same point in the
  subsurface.}

%% 
 % hyperbolic moveout 
 %%
The complication is that the signal from a given reflector arrives at
various receivers at different offsets at different times. This is due
to the fact that the waves propagate over larger distances in the
subsurface to account to the different offsets. Assuming that we
observe a reflector located at depth $z$ in a medium characterized by
the constant velocity $v$ with source and a receiver separated by
distance $2h$, then the total propagation time is:
%
\beq
t = \frac{d}{v} = \frac{2\sqrt{z^2+h^2}}{v} \;.
\eeq
%
The traveltime $t$ increases with the half-offset $h$ following a
hyperbolic curve, i.e. the traveltime is characterized by
\textbf{hyperbolic moveout}. Ideally, we would like to correct the
traveltime such that the signal from the given reflector appears at
the same time, regardless of offset. In this case, we can simply sum
all traces over offset and obtain data with higher SNR corresponding
to the factor $\sqrt{n}$, where $n$ is the number of traces in the
CMP. For example, we can correct all traces such that the traveltimes
for different offsets are equal to the traveltime at $h=0$, i.e.
%
\beq
t_0 = \frac{2z}{v}\;.
\eeq
% 
This process, simply called \textbf{Normal Moveout} (NMO)
\textbf{correction}, is one of the most basic imaging techniques used
for seismic data processing.

%% 
 % stacking velocity 
 %%
The NMO correction requires that we know the velocity $v$ which
describes the data hyperbolic trajectory as a function of time and
offset. This hyperbolic trajectory depends on the time at zero offset
$t_0$, and it is characterized by just one parameter which is called
the \textbf{stacking velocity}. This parameter has units of velocity,
and it is related to the medium velocity. If the medium changes as a
function of depth, then the stacking velocity for an event at a given
depth is simply a combination of all velocities above this
depth. Therefore, the stacking velocity does not necessarily have a
physical meaning other than that it characterizes the hyperbolic
moveout of reflections in CMPs. This velocity does not correspond to
any particular point in the subsurface, but it simply allows us to
bring all traces in a CMP to a common time.

%% 
 % picking
 %%
In order to apply the NMO correction, we need to know the stacking
velocity. We can find this parameter by scanning over a wide range of
possible stacking velocities and evaluating how strong the stack is at
every time. The velocity for which the stack is strongest is the
optimal parameter for the NMO correction. We can thus apply the
moveout correction as a function of offset, and then accumulate data
over offset to obtain the data stack.

%% 
 % NMO + stack 
 %%
The NMO-stack process consists of the following steps:
\begin{enumerate}
\item Sort data in common-midpoint gathers.
\item Find the stacking velocity by scanning.
\item Apply the normal moveout correction.
\item Stack over offset.
\end{enumerate}
The result of this operation is a \textbf{stack section}, i.e. a
representation of the data as a function of time and position (which
corresponds to the midpoint coordinates). This stack is comparable
with the \textbf{zero-offset section}, since all traces are corrected
to the zero-offset traveltime. However, the stack is characterized by
higher SNR since it accumulates information from multiple offsets, in
contrast with the zero-offset section which contains only the traces
from the zero offset. Therefore the stack section is more suitable for
interpretation, since more events with geologic significance are
visible.

Note that the NMO-stack process assumes that all data traces are
characterized by hyperbolic moveout. However, this assumption is not
true of all events. For example, the direct waves from the source to
the receivers are not hyperbolic. However, the NMO correction remaps
all events in the data regardless of their moveout. The non-hyperbolic
events represent noise that should be removed from the data before the
NMO correction, if possible.
