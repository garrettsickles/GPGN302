\section{Introduction}

Depth imaging requires that we construct a model of the Earth as a
function of depth. So far, we can evaluated parameters as a function
of time, i.e. we have worked in the data space. From now on, we make a
distinction between the following spaces:
%
\begin{itemize}
\item We represent data in the \textbf{data space},
  e.g. time-position-offset.
\item We represent images in the \textbf{image space},
  e.g. depth-position(-offset).
\end{itemize}

Scanning over data in CMP gathers allows us to evaluate a parameter
characterizing the hyperbolic moveout of the data. This parameter has
units of velocity and is called \textbf{stacking velocity}, simply
because its purpose is to construct the best stack after normal
moveout (NMO) correction. This parameter depends on the Earth velocity
through the data moveout, but it is not the velocity at any particular
point in the Earth. Furthermore, this parameter evaluated at a certain
time depends on waves propagating through the entire overburden above
a certain reflector.

For depth imaging, we need to do two operations:
\begin{enumerate}
\item Convert the stacking velocity into a velocity characterizing
  each position in the Earth, i.e. construct an \textbf{interval
    velocity}.
\item Place this velocity at the appropriate depth in the Earth.
\end{enumerate}

%% 
 % RMS to interval velocity conversion
 %%
The relationship between stacking velocities and interval velocities
is fairly complex, but it can summarized by the following relation
which is applicable to a layered Earth:
%
\beq \label{eqn:INT}
v_n^2 = \frac{V_n^2 t_n - V_{n-1}^2 t_{n-1}}{t_n - t_{n-1}} \;.
\eeq
%
In \req{INT}, $v_n$ is the interval velocity in layer $n$, $V_n$ and
$V_{n-1}$ are the stacking velocities for reflectors $n$ and $n-1$,
and $t_n$ and $t_{n-1}$ are the zero-offset times for the reflectors
$n$ and $n-1$. Quantities $V$ and $t$ are measurable through the stack
and picking process discussed in an earlier lab.

Generally speaking, $v^2\oft$ is constructed roughly as the derivative
of $V^2\oft$. This means that if $V\oft$ is not a smooth function,
then $v\oft$ is even less smooth. In fact, any small oscillation of
the stacking velocity $V\oft$ leads to large variations of the
interval velocity $v\oft$. Therefore, in order to constrain the
interval velocity to moderate variations, we need to smooth the
stacking velocity, thus potentially degrading its accuracy.

%% 
 % time-to-depth conversion
 %%
At the end of the process described in \req{INT} we obtain an interval
velocity as a function of time. What we need is to place this velocity
function in the Earth, i.e. to convert interval velocity from time to
depth. The depth conversion operation is also fairly complex, because
we are transforming to depth a function of time using a mapping that
depends on the function itself -- a strongly non-linear process. The
interval velocity $v\oft$ is stretched or squeezed depending on its
own fluctuations. Therefore, if the time interval velocity $v\oft$ is
not smooth, then the depth interval velocity $v\ofz$ can be
unreasonably squeezed or stretched.

The process leading to the velocity parameters in the Earth is known
as \textbf{migration velocity analysis}. In this lab, we are
discussing a simple velocity analysis process based on strong
assumptions about the Earth, i.e. a layered model (laterally
invariant). This process is known as ``Dix inversion'', after the
person who invented it many decades ago. More sophisticated procedures
exist and can construct velocity models for much more complicated
geology. All such procedures are based on a solution to an Inverse
Problem, one way or another.
