\section{Introduction}

Seismic data are acquired redundantly, i.e. multiple seismic
experiments cover the same physical space, in order to increase the
signal and to reduce the noise. The signal-to-noise ratio (SNR) is an
important factor in evaluating the quality of the acquired seismic
data. The general idea behind SNR improvement with the number of
measurements is that the signal is repeatable, while the noise is
not. By accumulating measurements at the same position in space, we
are increasing the signal, while decreasing the noise.

Consider measurements represented by probability density functions
(PDF) $f\lp S \rp$, i.e. a distributions that can be characterized by
mean $\bar{S}$ and by variance $\sigma^2$. We can refer to the mean
$\bar{S}$ as the signal and to the standard deviation $\sigma$ as the
noise, therefore we can define the SNR as the ratio of the mean to the
standard deviation:
%
\beq
SNR = \frac{\bar{S}}{\sigma} \;.
\eeq
%
This relation assumes that the signal and the noise are uncorrelated,
that the signal strength is constant with measurement and that the
noise is random with zero mean and constant variance.

The question is how does the SNR change when we add $n$ random
variables, each characterized by a probability density function (PDF)
with given mean and standard deviation? In order to answer this
question, we make use of two important results from Statistics:
\begin{itemize}
\item The PDF characterizing the sum of several random variables is
  the convolution of their PDFs.
\item The PDF of the convolution of random variables is characterized
  by mean and variance equal to the sum of means and variances of the
  convolved PDFs.
\end{itemize}
In other words the PDF characterizing the sum of $n$ measurements has
a mean equal to the sum of the means, and a variance equal to the sum
of the variances. Therefore, the PDF characterizing the sum of $n$
measurements has the mean $n\bar{S}$ and the variance $n\sigma^2$,
i.e. the standard deviation is $\sqrt{n} \sigma$. Thus, the SNR is
%
\beq
SNR_n = \frac{n\bar{S}}{\sqrt{n}\sigma}
      = \frac{\bar{S}}{\sigma} \sqrt{n}
      = SNR \sqrt{n} \;.
\eeq
%
In other words, the SNR improves proportionally with the square-root
of the number of measurements. For example, increasing the SNR by a
factor of $2$ requires $4$ measurements, increasing the SNR by a
factor of $5$ requires $25$ measurements,
etc. \rfgs{sig-00}-\rfn{dat-39} and \rfg{stk} illustrate this idea. 

% ------------------------------------------------------------
\inputdir{snr}
% ------------------------------------------------------------
\multiplot{5}{sig-00,sig-09,sig-19,sig-29,sig-39,dat-00,dat-09,dat-19,dat-29,dat-39}
{width=0.175\textwidth}{Signal (top row) and data (bottom row) obtained
  by adding bandlimited noise to the signal.}

\sideplot{stk}{width=0.5\textwidth}{Stack signal as a function of the
  number of traces.}

% ------------------------------------------------------------
\inputdir{XFig} 
% ------------------------------------------------------------
\sideplot{cmp}{width=\textwidth}{Common-midpoint gather geometry
  highlighting repeat measurements of the same point in the
  subsurface.}

The quantity we use to evaluate the number of repeat measurements is
\textbf{fold}, which exploits measurement redundancy in
common-midpoint gathers, i.e. subsets of the data selected for a fixed
midpoint, for all offsets $\DD\lp \mm=fixed, \ho, t \rp$.  \rFg{cmp}
shows the geometry of multiple source-receiver combinations with the
same midpoint. All rays connecting corresponding sources and receivers
sample the same position in the subsurface. Under the restrictive
assumption that the reflector is horizontal and that the medium is
laterally invariant, the traces corresponding to all offsets can be
stacked to generate one trace corresponding the midpoint
position. This addition does not make sense if the assumptions about
the medium are violated. However, we can still use the fold concept to
get a general idea of the redundancy of data acquisition.

In order to evaluate the fold, we need to convert the acquisition
parameters, i.e. source coordinates and offset, into midpoint and
half-offset coordinates using the relations
% 
\bea
\mm &=& \ss + \frac{\oo}{2} \\
\ho &=&       \frac{\oo}{2} \;.
\eea
% 
We can immediately note that the midpoint sampling is half the offset
sampling, i.e. the distance between receivers determines the
theoretical sampling along the reflector under investigation.
Once we have constructed the fold map in $\mm-\ho$ coordinates, we can
simply count how many times we hit the same midpoint regardless of
offset, and obtain the fold as a function of space.


