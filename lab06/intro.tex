\section{Introduction}

%% 
 % data
 %%
Seismic data are acquired with redundant geometry, i.e. we try to
illuminate a certain point in the subsurface from different
directions. We refer to such information as \textbf{prestack
  data}. These data are sorted in many domains and can be processed to
increase the signal-to-noise ratio using different procedures,
e.g. normal moveout correction and stack, thus leading to
\textbf{poststack data}. The stacked data are kinematically equivalent
with zero-offset data, i.e. an experiment for which the sources and
receivers are co-located, but have higher signal-to-noise ratio.

%% 
 % migration
 %%
Poststack migration produces \textbf{images} (i.e. reflectors) of the
Earth's interior as a function of depth. This is in contrast with
\textbf{data} which represent reflections as a function of time. 
%
The transformation from the data to the image is known as
\textbf{seismic migration} and requires knowledge of the velocity
model. We can produce velocity models using techniques like the ones
discussed in the preceding labs, i.e. build the stacking velocity,
then the interval velocity, etc. Such a process is known by the
generic name of \textbf{migration velocity analysis}. If the velocity
is not completely accurate, then the images are partially distorted,
thus indicating the need for a velocity update. Here we address the
following questions: how does the velocity influence the image and how
do we know if the velocity is correct?

%% 
 % exploding reflector model 
 %%
We can answer these questions using zero-offset migration under the
\textbf{exploding reflector} model, \rfg{expref}. This conceptual
model is designed to capture the behavior of waves corresponding to
sources and receivers located at the same position on the surface. One
possible scenario under which a wave leaving a source returns at the
same position requires that the angle of incidence on the reflector is
$90^\circ$. In this case, the reflected wave leaves the reflector in
opposite direction relative to the incoming wave, and thus propagates
along the same path to the receiver which is co-located with the
source. In this case, the down-going and up-going paths are identical
and the propagation times from the source to the reflector and from
the reflector to the receiver are also identical.\footnote{It is
  possible to observe data at zero offset even if the propagation
  paths are not identical. For example, waves can propagate downward
  and upward on opposite sides of a low velocity lens.} Thus, the time
at which we observe a reflection is precisely twice the propagation
time from the reflector to the receivers. This is true for all points
in the subsurface, therefore we can simulate zero offset data by
considering that the reflectors are populated with many sources which
are all triggered at the same time.

% ------------------------------------------------------------
\inputdir{XFig} 
\plot{expref}{width=\textwidth}{Exploding reflector model.}
% ------------------------------------------------------------

%% 
 % poststack migration
 %%
Assume that a point in the subsurface (e.g. a diffractor) becomes a
source at some time. Later we observe waves (diffractions) arriving at
the surface. We can find the point that is responsible for a
diffraction by running time backward until the moment when the
reflector exploded. If the velocity model is correct, then all the
diffracted energy focuses at the sources distributed along the
reflectors. Under the exploding reflector model, we can exploit this
idea in two ways: either we halve the observation time and then
propagate the data backward with the actual velocity until time zero,
or we keep the observation time and propagate the data backward in
time with the velocity halved. In both scenarios we move the data from
the surface to the actual reflector position, i.e. we perform
\textbf{poststack migration}.

%% 
 % post-stack migration procedure
 %%
Exploiting this general idea requires our ability to perform two
operations. First, we need to isolate diffractions from seismic data
and, second, we need to run time backward to allow waves to focus at
the source (i.e. to migrate the data).
\begin{itemize}
\item The first operation can be implemented using \textbf{plane-wave
    destruction}. We can assume that the data consist of nearly-planar
  objects, i.e. reflectors, and non-planar events,
  i.e. diffractions. We can also assume that reflections are stronger
  in magnitude than diffractions. If we can somehow destroy the
  strongest component of the data, i.e. the reflections, what is left
  is likely to be dominated by diffractions. We can then use the
  diffractions to assess the quality of the velocity model by
  evaluating focusing at the initial trigger time.

\item The second operation is \textbf{wave-equation migration}, which
  allows us to run time backward and focus waves at their
  sources. Since we don't know the correct model of the earth, we can
  try migration with different velocities and then select the one
  which is best, as measured by the quality of diffraction
  focusing. Once we find the correct model, then we can migrate the
  entire dataset containing both reflections and diffractions.
\end{itemize}

%% 
 % example
 %%
The example in \rfgs{vel00}-\rfn{vel01} demonstrates poststack imaging
under the exploding reflector model.  According to the Huygens
Principle, the data observed on the surface, \rfgs{dat00}-\rfn{dat01},
are the superposition of the wavefield from each individual source,
\rfgs{wfl00-01}-\rfn{wfl01-13}. Then, if we back-propagate the
observed data with correct velocity, we obtain the images in
\rfgs{zoi00-02}-\rfn{zoi01-02}, showing that the energy from each
source located on the reflector is correctly focused. However, if we
image the same data shown in \rfgs{dat00}-\rfn{dat01} using different
velocities, we obtain images with partially defocused sources on the
reflector, \rfgs{zoi00-00}-\rfn{zoi01-04}. This partial defocusing is
only visible if we can isolate from the data the energy corresponding
to a specific source on the reflector, i.e. the energy associated with
a diffractor due to a truncation or to a small-scale geologic feature.

% ------------------------------------------------------------
\inputdir{syncline} 
% ------------------------------------------------------------

\multiplot{2}{vel00,vel01,dat00,dat01}{width=0.45\textwidth}{(a)-(b)
  Syncline models with overlay depicting reflector, and (c)-(d) data
  acquired on the surface. }

\multiplot{2}{wfl00-01,wfl01-01,wfl00-05,wfl01-05,wfl00-09,wfl01-09,wfl00-13,wfl01-13}
{width=0.45\textwidth}{Wavefield snapshots generated at different
  times under the exploding reflector model.}

\multiplot{2}{zoi00-00,zoi01-00,zoi00-02,zoi01-02,zoi00-04,zoi01-04}
{width=0.45\textwidth} {Zero-offset images constructed for (a)-(b)
  low, (c)-(d) correct and (e)-(f) high migration velocity.}
