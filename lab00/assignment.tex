\section{Assignment}

\begin{enumerate}
\item Change to the example directory \texttt{lena} and build a few
  preset figures. Use the command \\
  \texttt{scons view} \\
  to look at the pictures on screen. Press \texttt{Q} to close a
  figure on screen. The program \texttt{scons} will continue to build
  figures in the order outlined in the \texttt{SConstruct}. When you
  are ok with your figures, use the command \\
  \texttt{scons lock} \\
  to move your figures to the archive -- this is necessary for
  reproducibility.
\item Return to the directory containing the file \texttt{assignment.tex}
  and edit this file. If necessary, remove the comment sign (\%) from
  the lines indicating the figure directory and the two figures to be
  included in the document:
  \begin{verbatim}
  \inputdir{lena}
  \multiplot{2}{lena,bpas}{width=0.45\textwidth}
  {(a) Input data, and (b) reproducible processing.}
  \end{verbatim}
  Your document now contains reproducible figures.
\item Rebuild the document using the command \\
  \texttt{scons read}.
\item Return to the \texttt{lena} directory and open the file called
  \texttt{SConstruct}. This file contains the rules for making the two
  figures. Lena's image is a standard benchmark for digital signal
  processing -- we will use it just as an image to play with
  processing rules. Processing is done using the function
  \texttt{Flow()} and results are created using the function
  \texttt{Result()}. Modify something in this \texttt{SConstruct}, for
  example the low pass filter applied to the image or the color scheme
  used for the plots, or the smoothness of the plots, etc.
  % 
  You could, for example, smooth the figure using a rule like
  \begin{center}
    \texttt{Flow('lsmo','lena','smooth rect1=10 rect2=10')}
  \end{center}
  This command will produce a new data file. You can clone the result
  rule used for the figure called \texttt{lena} to add a new result.
  % 
  You could also add color to the figure using a rule like
  \begin{center}
    \texttt{Result('lcol','lena',wplot.igrey2d('bias=128 color=j',par))}
  \end{center}
  
\item Rebuild the figures using the command \\
  \texttt{scons view}, \\
  then run \\
  \texttt{scons lock} \\
  when you are happy with your figures. Then, return to the lab
  directory and rebuild your document using the command \\
  \texttt{scons read}. \\
  Your new figures should be included in the document. You have just
  reproduced and modified a simple document. We will repeat this
  process in all following labs.
\end{enumerate}

% ------------------------------------------------------------
\inputdir{lena}
\multiplot{2}{lena,bpas}{width=0.45\textwidth}
{(a) Input data, and (b) reproducible processing.}

%\multiplot{2}{lsmo,lcol}{width=0.45\textwidth}
%{(a) Smoothed image, and (b) image with color.}
