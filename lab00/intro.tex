\section{Introduction}

The cornerstone of Science is the idea that any experimental result
can and should be reproduced by independent researchers. This is the
approach taken in experimental sciences (like Physics, Chemistry or
Biology), but it is less so in computational sciences. In this class,
we will do reproducible numeric experiments and produce reproducible
documents, like the one you are reading now.

We will make use of the following software systems:
\begin{itemize}

\item \LaTeX, available as free software from
  \texttt{http://www.latex-project.org}.  LaTeX is a high-quality
  typesetting system; we will use this system to write reproducible
  scientific documents.

\item \texttt{scons}, available as free software from
  \texttt{http://www.scons.org}.  \texttt{scons} (a dialect of Python)
  is a cross-platform software construction tool which substitutes the
  classical \texttt{make} utility; we will use this system to control
  numeric experiments and the reproducibility rules.

\item \mg, available as free software from
  \texttt{http://www.ahay.org}.  \mg is a software package for
  multidimensional data analysis and reproducible computational
  experiments; we will use this system to simulate and process seismic
  data.

\end{itemize}
I expect everyone to have a basic understanding of \texttt{Linux} --
an open-source software platform on which we will execute all numeric
experiments. I also expect everyone to be familiar with a basic text
editor, like \texttt{vi} or \texttt{gvim} or \texttt{emacs} or
\texttt{gedit}, etc.

The basic structure of a lab is the following:
\begin{verbatim}
  lab/SConstruct
  lab/paper.tex
  lab/intro.tex
  lab/assignment.tex
  lab/example1/SConstruct
  lab/example2/SConstruct
\end{verbatim}
The lab directory contains the files used to make the reproducible
documents, e.g. \texttt{paper.tex}, \texttt{intro.tex} and
\texttt{assignment.tex}. The example directories contain the rules for
numeric experiments documented in files named
\texttt{SConstruct}. These control the program \texttt{scons} which
runs \mg programs in sequence. For all labs, the usual flow of
operations is the following:
%
\begin{enumerate}
%
\item Setup the lab using the information provided in the handout.
%
\item Go to the lab directory and edit the file \texttt{paper.tex} to
  add your name, remove the setup content and add your answers in the
  file \texttt{assignment.tex}.
%
\item Go to the example directories to build figures documenting your
  numeric experiments -- here you edit the file \texttt{SConstruct}
  and run some of the following commands:
%
  \begin{itemize} 
    % 
  \item \texttt{scons} -- will generate all reproducible results by
    running programs in order. No figure will be displayed on screen
    during this execution. At the completion of this command, all
    figures will be available in the \texttt{Fig} directory. Use this
    command for programs that take a long time to execute.
    % 
  \item \texttt{scons view} -- is similar to the simple \texttt{scons}
    command, except that figures are displayed on screen as soon as
    they are created. Use this command while you develop and/or debug
    your scripts.
    % 
  \item \texttt{scons blah.view} -- is similar to \texttt{scons view},
    except that only one result, named \texttt{blah}, is produced and
    displayed on screen as soon as processing is complete.
    % 
  \item \texttt{scons lock} -- is similar to the simple \texttt{scons}
    command, except that all reproducible figures are ``locked away''
    at the completion of the run. Use this command to prepare figures
    to be included in the reproducible documents built with \LaTeX.
    % 
  \item \texttt{scons -c} -- will clean all temporary files. Use this
    command at the end of the lab to clean-up your directory and avoid
    filling-up your disk quota.
    % 
  \end{itemize}
%
\item Return to the lab directory and build your document using the
  commands \texttt{scons paper.read} or \texttt{scons read}. This will
  build a PDF document and open it on screen using a PDF reader. This
  is the document you will submit at the completion of the lab.
\end{enumerate}

All files named \texttt{SConstruct} in the example directories make
use or Python-like functions. The most common functions are the
following:
\begin{itemize}
\item \texttt{Flow} -- used to control the input and output to various
  processing commands. The usual syntax is:
  \begin{center}
    \texttt{Flow(output,input,command)}
  \end{center}

\item \texttt{Result} -- use to generate plots which are either
  displayed on screen using the command \texttt{scons view} or which
  are ``locked away'' using the command \texttt{scons lock}. The usual
  syntax is
  \begin{center}
    \texttt{Result(result,input,command)}
  \end{center}
  The figure \texttt{result} is available in the \texttt{Fig}
  directory. Sometimes the name of the input file is the same as the
  name of the result figure. In this case, the command could be
  \begin{center}
    \texttt{Result(result,command)}
  \end{center}
  
\item \texttt{Plot} -- used to generate temporary plots which could be
  overlain with other plots, or combined to form results.  The usual
  syntax is
  \begin{center}
    \texttt{Plot(plot,input,command)}
  \end{center}
  The figure \texttt{plot} is available in the current working
  directory.  Sometimes the name of the input file is the same as the
  name of the plot. In this case, the command could be
  \begin{center}
    \texttt{Plot(plot,command)}
  \end{center}

\end{itemize}

All \mg data are regularly-sampled multidimensional hypercubes, and
the data files have extension \texttt{.rsf}. All figures produced
using \mg~are in a format called \texttt{vplot} and the files have the
extension \texttt{.vpl}. Once a figure is generated, it can be
visualized using the program \texttt{xtpen}.  The usual syntax is:
  \begin{center}
    \texttt{xtpen file1.vpl [file2.vpl \dots]}
  \end{center}
 
