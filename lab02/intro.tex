\section{Introduction}

A complex medium is responsible for a complex seismic wavefield, and
thus for complex seismic data. By definition, a \textbf{wavefield} is
a mathematical representation of a propagating wave at all times $t$
and all positions in space $\xx=\{ x,y,z \}$:
%
\[ \UU\ofxt \;. \]
%
Similarly, \textbf{data} represents the wavefield at all times
restricted in space to the receiver coordinates $\rr=\{ \rx, \ry \}$:
%
\[ \DD\ofrt \;. \]
%
An \textbf{image} of the subsurface is a representation of the
reflectors as a function of space:
%
\[ \RR\ofx \;. \]
%
Data are normally much more complicated than images, therefore
geologic interpretation should be done on migrate images, instead of
data.

Various components of the seismic wavefield are labeled with different
names for easier understanding and communication. We can distinguish
the following types of waves:
\begin{itemize}
\item \textbf{direct waves} -- propagate from the source directly to
  the receivers, without bouncing of any interface in the subsurface;
  a special kind of direct waves are the so-called \textbf{diving
    waves};
\item \textbf{reflections} -- propagate from the source to sharp but
  continuous interfaces in the subsurface, then return to the
  receivers;
\item \textbf{diffractions} -- propagate from the source to sharp and
  discontinuous objects, then return to the receivers;
\item \textbf{multiples} -- propagate from the source in the
  subsurface like reflection or diffractions, but change direction
  more than once either on the surface or at subsurface
  discontinuities.
\end{itemize}

The seismic data are records of the ground motion as a function of
time. Observations at every receiver position are collected in a
seismic trace, $\DD\oft$. We often record a large number of traces for
different combinations of source and receiver coordinates, or other
derived coordinates:
\begin{itemize}
  \item $\ss=\{ \sx, \sy \}$: source coordinates;
  \item $\rr=\{ \rx, \ry \}$: receiver coordinates;
  \item $\ho=\{ \hx, \hy \}$: source-receiver half-offset;
    \[ \ho = \frac{\rr - \ss}{2} \]
  \item $\oo=\{ \ox, \oy \}$: source-receiver offset;
    \[ \oo = 2\ho\]
  \item $\mm=\{ \mx, \my \}$: source-receiver midpoint;
    \[ \mm = \frac{\rr + \ss}{2} \]
\end{itemize}
We can represent the seismic data either as a function of source and
receiver coordinates
%
\[ \DD\lp \ss,\rr,t \rp \;, \]
%
or as a function of midpoint and (half) offset coordinates
%
\[ \DD\lp \mm,\ho,t \rp \;. \]
%
This is just a change of coordinates -- the same traces are present in
both representations. For convenience, seismic observations are often
analyzed in subsets of the data:
\begin{itemize}
\item \textbf{common-shot gather} -- collection of all traces
  corresponding to the same source, \rfg{csg}:
  % 
  \[ \DD\lp \ss=fixed, \rr, t \rp \;.\]
  % 
\item \textbf{common-receiver gather} -- collection of all traces
  corresponding to the same receiver, \rfg{crg}:
  % 
  \[ \DD\lp \ss, \rr=fixed, t \rp \;.\]
  % 
\item \textbf{common-midpoint gather} -- collection of all traces with
  the same midpoint between sources and receivers, \rfg{cmp}:
  % 
  \[ \DD\lp \mm=fixed, \ho, t \rp \;.\]
  % 
\item \textbf{common-offset gather} -- collection of all traces with
  the same source-receiver separation, \rfg{cog}:
  % 
  \[ \DD\lp \mm, \ho=fixed, t \rp \;.\] 
  % 
  A special case of a common-offset section is a \textbf{zero-offset
    section}, i.e. $\ho=\mathbf{0}$. Another special case of a
  common-offset section is a \textbf{near-offset section},
  i.e. $\ho=small$.
\item \textbf{time slice} -- collection of all samples with the same
  recording time:
  % 
  \[ \DD\lp \ss, \rr, t=fixed \rp \;,\]
  % 
  or 
  % 
  \[ \DD\lp \mm, \ho, t=fixed \rp \;.\]
  % 
\end{itemize}

% ------------------------------------------------------------
\inputdir{XFig}
% ------------------------------------------------------------
\multiplot{2}{csg,crg,cmp,cog}{width=0.45\textwidth}
{(a) Common-shot gather, (b) common-receiver gather, (c)
  common-midpoint gather, and (d) common-offset gather.}

Seismic images can be constructed by processing the data in different
subsets. For example, we can consider the following cases:
\begin{itemize}
\item \textbf{Zero-offset imaging} -- we construct an image of the
  reflectivity in the Earth using a common-offset dataset:
  % 
  \[ \RR_z \ofx \]
  % 
  This type of imaging corresponds to the passive-array setup
  discussed earlier.
\item \textbf{Shot-record imaging} -- we construct an image of the
  reflectivity in the Earth using a common-shot gather:
  % 
  \[ \RR_s \ofx \]
  % 
  This type of imaging corresponds to the active-array setup discussed
  earlier. A typical dataset consists of multiple shot-gathers, each
  having a different position in space. Each shot-gather generates an
  independent image which is a different view of the interior of the
  Earth. The final image is the summation of the images constructed
  for individual shot-records:
  % 
  \beq 
  \RR\ofx = \esum{shots} \RR_s \ofx 
  \eeq
  % 
\end{itemize}
In both cases, imaging requires that we know the parameters that
control wave propagation in the Earth, i.e. the velocity and density
for an acoustic seismic wavefield.
